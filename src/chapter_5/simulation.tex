\documentclass[../../main.tex]{subfiles}

\begin{document}

\setchapterpreamble[u]{\margintoc}

\chapter{模拟}

模拟问题考验的是通过对具体问题的分析去找到循环不变量,或者按照数学归纳法的思想以及从特殊
到一般的思想去解决问题。

\section{加减乘除问题}

\subsection{\href{https://leetcode.cn/problems/add-two-numbers/}{两数相加}}

这个题目是一个非常简单的题目,我们只需要按照加法的定义去实现即可。但是,我们需要注意的是,我们
需要考虑进位的问题。因此,我们需要使用一个变量来记录进位。同时,我们可能遇到两个链表长度不一致
的情况,因此我们需要在循环中判断两个链表是否为空。然而,我们这样会让代码变得十分冗长。因此,我们
可以使用一个小技巧,如果链表为空,我们就让其值为0,这样我们就可以简化我们的代码了。

\lstinputlisting[language=C++]{code/add-two-numbers.cpp}

\subsection{\href{https://leetcode.cn/problems/add-binary/}{二进制求和}}

这个题目和上一个题目的思路是一样的,没有任何区别。除了把链表换成了字符串而已。这个问题更关键的
在于不应该动态分配内存,而是创建一个新的字符串来存储结果。新的字符串的大小应该是两个字符串中
较长的那个字符串的长度加1或者相等。

\lstinputlisting[language=C++]{code/add-binary.cpp}

\begin{kaobox}[title=类似题目]
  \begin{itemize}
    \item \href{https://leetcode.cn/problems/add-strings/}{字符串相加}
    \item \href{https://leetcode.cn/problems/add-to-array-form-of-integer/}{加一}
    \item \href{https://leetcode.cn/problems/add-to-array-form-of-integer/}{数组形式的整数加法}
  \end{itemize}
\end{kaobox}

\subsection{\href{https://leetcode-cn.com/problems/multiply-strings/}{字符串相乘}}

任何的相乘问题都可以转化为加法问题。显然,我们已经在前面的题目中解决了字符串相加的问题。也就是两数之和。
因此,我们只需要把这个问题转化为多次两数之和的问题即可。我们可以使用一个数组来存储每一次的结果,然后
再把这些结果相加即可\sidenote{这个做法显然是要超时的,因为这样复杂度会相当高,而且我们必须要处理溢出的
情况。需要进行非常复杂的思考}。

另一个思路就是利用分治的方法去解决这个问题,这个思路很简单的。我们假设$num_{1}$为偶数:

\begin{align*}
  num_{1} \times num_{2} &= (num_{1} / 2 ) * (num_{2} + num_{2}) \\
                         &= (num_{1} / 4) * (num_{2} + num_{2} + num_{2} + num_{2}) \\
\end{align*}

这样我们就可以实现以复杂度为$O(\log n)$的算法了。其中,$n$为$num_{1}$的大小。然而,对于这个题我们
也很难用这个方法去解决,因为我们仍然需要处理数据溢出的问题。

思考到这儿,我们就必须回到最原始的方式,还记得我们在小学的时候是怎么做乘法的吗\sidenote{
  $$
  \begin{array}{c@{}c@{}c@{}c@{}c}
    & & & 3 & 4 \\
    & & \times & 2 & 5 \\
    \cline{2-5}
    & & 7 & 0 & \\
  + & 6 & 8 & & \\
  \cline{1-5}
    & 8 & 9 & 0 & \\
  \end{array}
  $$
}?我们就可以按照这个思路来解决这个问题。由于我们可能需要计算很多次的重复操作,故使用一个哈希表来记录
每一次的结果,这样我们就可以避免重复计算了。

\lstinputlisting[language=C++]{code/multiply-strings.cpp}

\section{螺旋矩阵系列}

\subsection{\href{https://leetcode.cn/problems/spiral-matrix/}{螺旋矩阵}}

为了实现螺旋矩阵的遍历,我们必须通过画图的方式去找到循环不变量。如下图所示,我们给出了一个
$6 \times 6$的矩阵。其中红色是第一次遍历的路径,蓝色是第二次遍历的路径,绿色是第三次遍历的路径。
我们可以发现,每一次遍历都是从左上角开始,然后向右走,然后向下走,然后向左走,然后向上走。故我们
可以维持三个变量,一个为\texttt{i}代表当前位于的行和列,一个是\texttt{widthIndex}代表当前的
右边界,一个是\texttt{heightIndex}代表当前的下边界。

\begin{tikzpicture}[]

  \draw[arrow, color=red] (0,5) -- (5, 5);
  \draw[arrow, color=red] (5,5) -- (5, 0);
  \draw[arrow, color=red] (5,0) -- (0, 0);
  \draw[arrow, color=red] (0,0) -- (0, 4);

  \draw[arrow, color=blue] (1,4) -- (4, 4);
  \draw[arrow, color=blue] (4,4) -- (4, 1);
  \draw[arrow, color=blue] (4,1) -- (1, 1);
  \draw[arrow, color=blue] (1,1) -- (1, 3);

  \draw[arrow, color=green] (2,3) -- (3, 3);
  \draw[arrow, color=green] (3,3) -- (3, 2);
  \draw[arrow, color=green] (3,2) -- (2, 2);

  \foreach \x in {0,...,5} {
    \foreach \y in {0,...,5} {
      \filldraw (\x,\y) circle (0.1);
    }
  }
\end{tikzpicture}

同时,我们必须考虑当\texttt{i = widthIndex}即上边界与下边界重合以及\texttt{j = heightIndex}的情况,
即左边界和右边界重叠的情况。这样我们可以得出如下的代码:

\lstinputlisting[language=C++]{code/spiral-matrix.cpp}

\subsection{\href{https://leetcode.cn/problems/spiral-matrix-ii/}{螺旋矩阵 II}}

这个题与上一个题没有任何本质的区别,这个题只不过限定了矩阵为方阵,然而上述的代码是通用的,因此我们只需要
维护一个变量,上面的代码直接套用。

\lstinputlisting[language=C++]{code/spiral-matrix-ii.cpp}

\section{字符串问题}

字符串问题有许多关于模拟的问题,这些问题都是非常容易理解的问题,但是我们需要注意的是,我们需要考虑
边界条件,以及特殊情况。

\subsection{\href{https://leetcode.cn/problems/simplify-path/}{简化路径}}

如何简化路径了,首先我们需要提取出路径中的每一个部分。我们可以简单地发现每一个路径都是由斜杠分割的。
我们可以发现如果我们按照斜杠来分割路径,我们可以得到如下三种情况的路径:

\begin{itemize}
  \item \texttt{""},即空路径,这个路径是无效的,我们需要忽略掉;
  \item \texttt{"."},即当前路径,这个路径是无效的,我们需要忽略掉;
  \item \texttt{".."},即上一级路径,这个路径是有效的,我们需要修改路径;
  \item \texttt{其他},即其他路径,这个路径是有效的,我们需要修改路径;
\end{itemize}

然后我们就可以写出如下的代码:

\lstinputlisting[language=C++]{code/simplify-path.cpp}

\subsection{\href{https://leetcode.cn/problems/decode-string/}{字符串解码}}

显然这是一个递归问题,如果你写过解释器就能够很简单地解决这个问题。当我们遇到了一个数字的时候
我们需要记录这个数字,然后当我们遇到了一个左方括号的时候,我们需要递归地解码这个字符串,然后
遇到右方括号的时候,返回这个字符串即可。

\lstinputlisting[language=C++]{code/decode-string.cpp}

\section{树模拟问题}

\subsection{\href{https://leetcode.cn/problems/populating-next-right-pointers-in-each-node/}
{填充每个节点的下一个右侧节点指针}}

这个题我们如果使用层次遍历直接就秒杀了。但是我们可以完全从另一个角度出发解决这个问题。由于限定
了树为完全二叉树,我们可以很明确,每个节点必然有一个左孩子和一个右孩子。对于每一层,我们拥有
上一层的信息,也就是上一层本质是一个链表,我们可以依赖这个信息对其孩子们做处理。

很明显,有两种情况,连接同一个父节点的左孩子和右孩子。然后连接相邻父节点的右孩子和其左孩子。
不断如此操作即可得出代码。

\lstinputlisting[language=C++]{code/populating-next-right-pointers-in-each-node.cpp}

\subsection{\href{https://leetcode.cn/problems/populating-next-right-pointers-in-each-node-ii/}
{填充每个节点的下一个右侧节点指针II}}

本题与上题唯一的区别在于树变为了普通的二叉树了,我们仍然考虑我们维持了上一层的状态。但是,我们的
连接需要更加具有普遍性,也就是我们需要把连接变为链表的尾插法。这样我们就可以解决这个问题了。对于
每层我们都需要有一个虚拟节点作为其头节点。

\lstinputlisting[language=C++]{code/populating-next-right-pointers-in-each-node-ii.cpp}

\section{区间问题}

\subsection{\href{https://leetcode.cn/problems/merge-intervals/}{合并区间}}

首先我们对区间进行排序,按照左端点进行排序。这样对任意一个区间值$[n_{i}, m_{i}]$我们都可以有
$n_{0} < n_{1} < \dots < n_{i}$。对于两个相邻的区间$[n_{i}, m_{i}]$以及$[n_{i + 1}, m_{i + 1}]$
我们可以总结出如下三种情况:

\begin{itemize}
  \item $n_{i + 1} > m_{i}$,两个区间完全独立。
  \item $n_{i + 1} < m_{i}, m_{i + 1} \leq m_{i}$,区间变为$n_{i}, m_{i}$。
  \item $n_{i + 1} < m_{i}, m_{i + 1} > m_{i}$,区间变为$n_{i}, m_{i + 1}$。
\end{itemize}

于是,我们就可以写出如下的代码:

\lstinputlisting[language=C++]{code/merge-intervals.cpp}

\subsection{\href{https://leetcode.cn/problems/insert-interval/}{插入区间}}

假设我们当前遍历的区间为$[n, m]$,我们需要插入的区间为$[k, l]$。当$k > m$时,我们直接
将当前的区间加入答案。如果$k <= m$,那么证明我们发现了新的区间会与当前的区间造成重叠。
此时,我们需要对区间进行合并。

首先,我们需要确定这个区间的左端点,这个左端点必然是$n$和$k$中的较小值。然后,我们需要这个
区间的右端点,我们需要遍历找到某个区间,其左端点大于$l$。然后我们就可以知道这个区间的上一个
区间。我们需要确定这个区间的右端点,这个右端点必然是$l$和上一个区间的右端点中的较大值。

\lstinputlisting[language=C++]{code/insert-interval.cpp}


\section{特定模拟问题}

\subsection{\href{https://leetcode.cn/problems/next-permutation/}{下一个排列}}

下一个排列是极其重要的模拟问题,其可以用于生成全排列以及用于解决其他的一些问题。其算法如下:

\begin{itemize}
  \item 从右到左扫描排列,找到第一个比右边元素小的元素,记为$i$。
  \item 从右到左扫描排列,找到第一个比$i$大的元素,记为$j$。
  \item 交换$i$和$j$。
  \item 将$j \to n - 1$的元素逆序。
\end{itemize}

\lstinputlisting[language=C++]{code/next-permutation.cpp}

\begin{kaobox}[title=类似题目]
  \begin{itemize}
    \item \href{https://leetcode-cn.com/problems/next-greater-element-iii/}{下一个更大元素 III}
    \item \href{https://leetcode.cn/problems/permutation-sequence/}{排列序列}
  \end{itemize}
\end{kaobox}

\end{document}
