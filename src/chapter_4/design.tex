\documentclass[../../main.tex]{subfiles}

\begin{document}

\setchapterpreamble[u]{\margintoc}

\chapter{设计数据结构}

设计数据结构是一个在我看来比算法都要重要的能力,合理地考虑数据结构的设计,可以让我们的算法更加简洁,更加
高效。本章主要记录Leetcode中一些设计数据结构的题目。

\section{缓存类问题}

\subsection{\href{https://leetcode.cn/problems/lru-cache/}{LRU缓存}}
\label{subsec:lru-cache}

LRU缓存的思想十分简单,就是当缓存满了的时候,我们需要将最近最少使用的数据删除。因此我们需要一个数据结构
能够表示最少,一种简单的方式就是使用链表,链表的头部表示最近使用的数据,链表的尾部表示最近最少使用的数据。
当我们需要删除数据的时候,只需要删除链表的尾部即可。当我们需要插入数据的时候,只需要将数据插入到链表的头部
即可。如果数据已经存在,我们需要将数据移动到链表的头部。

然而,该题目还涉及到了数据的查找,因此我们需要一个数据结构能够快速地查找到数据。这里我们可以使用哈希表来
实现。

最棘手的是,哈希表应该保存什么值,仔细的一思考,我们应该存储链表的迭代器,这样我们就能快速访问其值。当我们
从链表移除最短使用的数据的时候,我们需要从哈希表中删除其迭代器,我们需要其键值,所以我们需要在链表中存储
其键值。

\begin{lstlisting}[language=C++,style=kaolstplain]
class LRUCache {
private:
  int size;
  list<pair<int, int>> items{};
  unordered_map<int, typename list<pair<int, int>>::iterator> keyToIterators{};
};
\end{lstlisting}

剩下的算法就十分简单了,可见数据结构的设计是多么的重要。

\lstinputlisting[language=C++]{code/lru-cache.cpp}

\subsection{LRU-K缓存}

尽管Leetcode上没有LRU-K缓存的题目,但我个人认为实现LRU-K是进一步实现LFU缓存的基础。LRU-K缓存
维护一个大小为$n$的缓存,其中每个缓存条目都有一个关联的计数器。每当一个缓存条目被访问时,它的计数器
就会增加1。在进行缓存淘汰时,算法会选择计数器最小的$K$个条目中最久未被使用的条目进行淘汰。

我们可以使用两个链表来实现LRU-K缓存,一个链表用于维护访问次数大于等于$K$的缓存条目,另一个链表用于
维护访问次数小于$K$的缓存条目。当需要淘汰缓存条目时,使用LRU淘汰访问次数小于$K$的链表中的条目,然后
再在访问次数大于等于$K$的链表中使用LRU算法淘汰最长时间未被使用的条目。因此,我们可以定义如下的
数据结构:

可以看见唯一的区别在于,我们需要处理当访问次数从小于$K$变为大于等于$K$的情况,这时我们需要将数据从
小于$K$的链表中移除,然后插入到大于等于$K$的链表中。并对大于等于$K$的链表进行LRU淘汰。同时,我们
需要更改。其余的情况,均可以使用LRU算法来处理\sidenote{
这个问题交给读者自己思考。此处仅仅定义基本的数据结构}。

\begin{lstlisting}[language=C++,style=kaolstplain]
class LRUKCache {
private:
  struct Data {
    int key;
    int value;
    int counter{1};

    Data() = default;
    Data(int key, int value) : key(key), value(value) {}
  };

  int size;
  int k;
  list<Data> itemsLessThanK{};
  list<Data> itemsGreaterOrEqualToK{};
  unordered_map<int, list<Data>::iterator> lessThanKIter{};
  unordered_map<int, list<Data>::iterator> greaterOrEqualKIter{};
};
\end{lstlisting}

\subsection{\href{https://leetcode-cn.com/problems/lfu-cache/}{LFU缓存}}

LFU缓存,淘汰最少使用的数据。然而,如何最少呢?我们可以对每一个频率维护一个链表,链表的头部表示最近
使用的数据,链表的尾部表示最近最少使用的数据。当我们需要删除数据的时候,只需要删除链表的尾部即可。当
我们需要插入数据的时候,只需要将数据插入到链表的头部即可。如果数据已经存在,我们需要将当前链表的节点
移动到下一个频率对应的链表的头部\sidenote{你可能或许有点明白为什么我要介绍LRU-K了。其算法虽然不一样,
但是实现的思想高度一致。}。

然后我们仍然需要思考如何有效地构建数据结构,首先是需要定义\texttt{info}的数据结构,其应该包含类似
于\ref{subsec:lru-cache}的数据结构。然后我们仍然可以类似定义哈希表。最关键的问题是我们需要保存
一系列的链表:

\begin{lstlisting}[language=C++,style=kaolstplain]
class LFUCache {
  private:
    class Info {
    private:
      int key;
      int value;
      int counter;

    public:
      Info() = delete;
      Info(int k, int v) : key{k}, value{v}, counter{1} {}

      int getKey() const { return key; }

      void setValue(int v) { value = v; }
      int getValue() const { return value; }

      int plusCounter() {
        counter++;
        return counter;
      }
    };

    vector<list<Info>> infos{};

    unordered_map<int, list<Info>::iterator> keysToIterators{};

    int size;
    int minFrequency{1};
};
\end{lstlisting}

实现的逻辑稍微复杂了一些,但是抓住本质就行了。我认为本质就是对链表进行移动。举个例子。

\begin{example}
  假设我们有如下的数据:

  \begin{itemize}
    \item count 1 : a -> b -> c.
    \item count 2 : d -> e.
    \item count 3 : ...
  \end{itemize}

  当我们使用\texttt{get(a)}时,我们需要将\texttt{a}从count 1的链表中移除,然后插入到count 2的链表。


  \begin{itemize}
    \item count 1 : b -> c.
    \item count 2 : a -> d -> e.
    \item count 3 : ...
  \end{itemize}

\end{example}

\lstinputlisting[language=C++]{code/lfu-cache.cpp}

\begin{kaobox}[title=类似题目]
  实际上,当你完成了这个题,你就能够十分简单地完成\href{https://leetcode.cn/problems/all-oone-data-structure/}{
    全 O(1) 的数据结构}
\end{kaobox}

\end{document}
