\documentclass[../../main.tex]{subfiles}

\begin{document}

\setchapterpreamble[u]{\margintoc}

\chapter{设计数据结构}

设计数据结构是一个在我看来比算法都要重要的能力,合理地考虑数据结构的设计,可以让我们的算法更加简洁,更加
高效。本章主要记录Leetcode中一些设计数据结构的题目。

\section{缓存类问题}

\subsection{\href{https://leetcode.cn/problems/lru-cache/}{LRU缓存}}
\label{subsec:lru-cache}

LRU缓存的思想十分简单,就是当缓存满了的时候,我们需要将最近最少使用的数据删除。因此我们需要一个数据结构
能够表示最少,一种简单的方式就是使用链表,链表的头部表示最近使用的数据,链表的尾部表示最近最少使用的数据。
当我们需要删除数据的时候,只需要删除链表的尾部即可。当我们需要插入数据的时候,只需要将数据插入到链表的头部
即可。如果数据已经存在,我们需要将数据移动到链表的头部。

然而,该题目还涉及到了数据的查找,因此我们需要一个数据结构能够快速地查找到数据。这里我们可以使用哈希表来
实现。

最棘手的是,哈希表应该保存什么值,仔细的一思考,我们应该存储链表的迭代器,这样我们就能快速访问其值。当我们
从链表移除最短使用的数据的时候,我们需要从哈希表中删除其迭代器,我们需要其键值,所以我们需要在链表中存储
其键值。

\begin{lstlisting}[language=C++,style=kaolstplain]
class LRUCache {
private:
  int size;
  list<pair<int, int>> items{};
  unordered_map<int, typename list<pair<int, int>>::iterator> keyToIterators{};
};
\end{lstlisting}

剩下的算法就十分简单了,可见数据结构的设计是多么的重要。

\lstinputlisting[language=C++]{code/lru-cache.cpp}

\subsection{LRU-K缓存}

尽管Leetcode上没有LRU-K缓存的题目,但我个人认为实现LRU-K是进一步实现LFU缓存的基础。LRU-K缓存
维护一个大小为$n$的缓存,其中每个缓存条目都有一个关联的计数器。每当一个缓存条目被访问时,它的计数器
就会增加1。在进行缓存淘汰时,算法会选择计数器最小的$K$个条目中最久未被使用的条目进行淘汰。

我们可以使用两个链表来实现LRU-K缓存,一个链表用于维护访问次数大于等于$K$的缓存条目,另一个链表用于
维护访问次数小于$K$的缓存条目。当需要淘汰缓存条目时,使用LRU淘汰访问次数小于$K$的链表中的条目,然后
再在访问次数大于等于$K$的链表中使用LRU算法淘汰最长时间未被使用的条目。因此,我们可以定义如下的
数据结构:

可以看见唯一的区别在于,我们需要处理当访问次数从小于$K$变为大于等于$K$的情况,这时我们需要将数据从
小于$K$的链表中移除,然后插入到大于等于$K$的链表中。并对大于等于$K$的链表进行LRU淘汰。同时,我们
需要更改。其余的情况,均可以使用LRU算法来处理\sidenote{
这个问题交给读者自己思考。此处仅仅定义基本的数据结构}。

\begin{lstlisting}[language=C++,style=kaolstplain]
class LRUKCache {
private:
  struct Data {
    int key;
    int value;
    int counter{1};

    Data() = default;
    Data(int key, int value) : key(key), value(value) {}
  };

  int size;
  int k;
  list<Data> itemsLessThanK{};
  list<Data> itemsGreaterOrEqualToK{};
  unordered_map<int, list<Data>::iterator> lessThanKIter{};
  unordered_map<int, list<Data>::iterator> greaterOrEqualKIter{};
};
\end{lstlisting}

\subsection{\href{https://leetcode-cn.com/problems/lfu-cache/}{LFU缓存}}

LFU缓存,淘汰最少使用的数据。然而,如何最少呢?我们可以对每一个频率维护一个链表,链表的头部表示最近
使用的数据,链表的尾部表示最近最少使用的数据。当我们需要删除数据的时候,只需要删除链表的尾部即可。当
我们需要插入数据的时候,只需要将数据插入到链表的头部即可。如果数据已经存在,我们需要将当前链表的节点
移动到下一个频率对应的链表的头部\sidenote{你可能或许有点明白为什么我要介绍LRU-K了。其算法虽然不一样,
但是实现的思想高度一致。}。

然后我们仍然需要思考如何有效地构建数据结构,首先是需要定义\texttt{info}的数据结构,其应该包含类似
于\ref{subsec:lru-cache}的数据结构。然后我们仍然可以类似定义哈希表。最关键的问题是我们需要保存
一系列的链表:

\begin{lstlisting}[language=C++,style=kaolstplain]
class LFUCache {
  private:
    class Info {
    private:
      int key;
      int value;
      int counter;

    public:
      Info() = delete;
      Info(int k, int v) : key{k}, value{v}, counter{1} {}

      int getKey() const { return key; }

      void setValue(int v) { value = v; }
      int getValue() const { return value; }

      int plusCounter() {
        counter++;
        return counter;
      }
    };

    vector<list<Info>> infos{};

    unordered_map<int, list<Info>::iterator> keysToIterators{};

    int size;
    int minFrequency{1};
};
\end{lstlisting}

实现的逻辑稍微复杂了一些,但是抓住本质就行了。我认为本质就是对链表进行移动。举个例子。

\begin{example}
  假设我们有如下的数据:

  \begin{itemize}
    \item count 1 : a -> b -> c.
    \item count 2 : d -> e.
    \item count 3 : ...
  \end{itemize}

  当我们使用\texttt{get(a)}时,我们需要将\texttt{a}从count 1的链表中移除,然后插入到count 2的链表。


  \begin{itemize}
    \item count 1 : b -> c.
    \item count 2 : a -> d -> e.
    \item count 3 : ...
  \end{itemize}

\end{example}

\lstinputlisting[language=C++]{code/lfu-cache.cpp}

\begin{kaobox}[title=类似题目]
  实际上,当你完成了这个题,你就能够十分简单地完成\href{https://leetcode.cn/problems/all-oone-data-structure/}{
    全 O(1) 的数据结构}
\end{kaobox}

\section{栈与队列}

\subsection{\href{https://leetcode-cn.com/problems/min-stack/}{最小栈}}

要求实现一个基本的栈,同时能够找到栈里面的最小的元素。我们可以使用两个栈来实现,一个栈用于存储数据,另一个
栈用于存储最小的元素。当我们需要插入数据的时候,我们需要判断当前的数据是否小于等于最小栈的栈顶元素,如果
是,则将数据插入到最小栈中。当我们需要删除数据的时候,我们需要判断当前的数据是否等于最小栈的栈顶元素,如果
是,则将最小栈的栈顶元素删除\sidenote{实际上,我并不认为这是一个容易想到的思路,本质上就是利用单调
栈的思路解决问题,我们需要记录在拓扑结构关系上的小于当前最小值的值。}。

\lstinputlisting[language=C++]{code/min-stack.cpp}

\subsection{\href{https://leetcode.cn/problems/implement-stack-using-queues/}{用队列实现栈}}

如何用队列实现栈了。我们没有任何好的办法。我们必须在一个元素新加入元素的时候,重新组织队列里面的元素。
我们利用数学归纳法寻找循环不变量,假设前$n$个元素在队列中满足栈的性质,当我们加入$n + 1$个元素时候,
第$n + 1$个元素应该最先出队,然而按照队列的性质,第$n + 1$个元素应该最后出队。因此我们首先将该元素
加入队列,然后把所有剩下的元素出队,再入队,这样就能够保证第$n + 1$个元素最先出队。

\begin{example}
  我们通过举一个例子,假设此时队列中的元素为\texttt{[1, 2, 3]},我们需要加入元素\texttt{4},此时
  队列中的元素应该为\texttt{[4, 1, 2, 3]},然而我们需要把队列中的元素变为\texttt{[1, 2, 3, 4]}。
  故我们先出队,再入队,即可。
\end{example}

\lstinputlisting[language=C++]{code/implement-stack-using-queues.cpp}

\subsection{\href{https://leetcode-cn.com/problems/implement-queue-using-stacks/}{用栈实现队列}}

反过来,我们又该如何用栈实现队列了。当我们用栈存储元素时,如果我们向获取队列的头部元素,我们需要找到栈的底部元素。
在这个过程中,我们可以使用另一个栈去保存当前栈的数据。

\begin{example}
  假设我们把元素\texttt{[1, 2, 3]}压入栈\texttt{st}中,此时栈的底部元素为1,我们需要获取队列的头部元素,
  我们需要把1弹出,在这个过程中,我们需要弹出栈顶元素3,我们把栈顶元素3压入到另一个栈\texttt{qu}中,然后
  弹出栈顶元素2,我们把栈顶元素2压入到另一个栈中,然后弹出栈顶元素1,我们就得到了队列的头部元素。
  同时,我们的另一个栈\texttt{qu}就维持了队列的顺序。
\end{example}

从上面的例子可以看出,当\texttt{qu}不为空时,我们能直接找到队列的头部元素,当\texttt{qu}为空时,我们需要
将\texttt{st}中的元素全部弹出,然后压入到\texttt{qu}中,这样我们就能够找到队列的头部元素了。经过上述的分析
我们可以写出如下的代码:

\lstinputlisting[language=C++]{code/implement-queue-using-stacks.cpp}

\subsection{\href{https://leetcode-cn.com/problems/design-circular-queue/}{设计循环队列}}

循环队列是十分常见的数据结构,其在许多计算机系统中都有应用\sidenote{循环队列主要是提供了一个十分重要的
抽象,也就是我们的队列可以接收无穷的元素,在TCP/IP协议中,RingBuffer的实现就与循环队列有关}。

要实现循环队列,我们只需要一个数组,两个指针,一个指针表示当前队列的头部,一个指针表示当前队列的尾部。
同时我们需要一个变量来记录当前队列的大小\sidenote{
你可能会疑惑,为什么我们需要记录当前队列的大小,而不是直接使用两个指针的关系来判断。这是因为我们根本
无法区分队列为空和队列为满的情况。
}。初始的时候,两个指针都指向数组的第一个元素。

\begin{tikzpicture}
  \def\width{1}
  \def\height{1}
  \def\array{{ , , , }}

  \foreach \x/\val in {0/, 1/, 2/, 3/} {
      \draw (\x*\width, 0) rectangle ++(\width, \height) node[midway] {};
  }

  \draw[arrow, color=red] (0.2, 1.5) -- (0.2, 1) node[midway, above left] {front};
  \draw[arrow, color=blue] (0.6, 1.5) -- (0.6, 1) node[midway, above right] {rear};
\end{tikzpicture}

\begin{tikzpicture}
  \def\width{1}
  \def\height{1}
  \def\array{{ , , , }}

  \draw (0, 0) rectangle ++(\width, \height) node[midway] {0};

  \foreach \x/\val in {1/, 2/, 3/} {
      \draw (\x*\width, 0) rectangle ++(\width, \height) node[midway] {};
  }
  \draw[arrow, color=red] (0.5, 1.5) -- (0.5, 1) node[midway, above left] {front};
  \draw[arrow, color=blue] (1.5, 1.5) -- (1.5, 1) node[midway, above right] {rear};
\end{tikzpicture}

\begin{tikzpicture}
  \def\width{1}
  \def\height{1}
  \def\array{{ , , , }}

  \draw (0, 0) rectangle ++(\width, \height) node[midway] {0};
  \draw (1, 0) rectangle ++(\width, \height) node[midway] {1};


  \foreach \x/\val in {2/, 3/} {
      \draw (\x*\width, 0) rectangle ++(\width, \height) node[midway] {};
  }

  \draw[arrow, color=red] (0.5, 1.5) -- (0.5, 1) node[midway, above left] {front};
  \draw[arrow, color=blue] (2.5, 1.5) -- (2.5, 1) node[midway, above right] {rear};
\end{tikzpicture}

\begin{tikzpicture}
  \def\width{1}
  \def\height{1}
  \def\array{{ , , , }}

  \draw (0, 0) rectangle ++(\width, \height) node[midway] {0};
  \draw (1, 0) rectangle ++(\width, \height) node[midway] {1};
  \draw (2, 0) rectangle ++(\width, \height) node[midway] {2};

  \foreach \x/\val in {3/} {
      \draw (\x*\width, 0) rectangle ++(\width, \height) node[midway] {};
  }

  \draw[arrow, color=red] (0.5, 1.5) -- (0.5, 1) node[midway, above left] {front};
  \draw[arrow, color=blue] (3.5, 1.5) -- (3.5, 1) node[midway, above right] {rear};
\end{tikzpicture}

\begin{tikzpicture}
  \def\width{1}
  \def\height{1}
  \def\array{{ , , , }}

  \draw (0, 0) rectangle ++(\width, \height) node[midway] {0};
  \draw (1, 0) rectangle ++(\width, \height) node[midway] {1};
  \draw (2, 0) rectangle ++(\width, \height) node[midway] {2};
  \draw (3, 0) rectangle ++(\width, \height) node[midway] {3};

  \draw[arrow, color=red] (0.2, 1.5) -- (0.2, 1) node[midway, above left] {front};
  \draw[arrow, color=blue] (0.6, 1.5) -- (0.6, 1) node[midway, above right] {rear};
\end{tikzpicture}

于是,我们可以得出如下的代码:

\lstinputlisting[language=C++]{code/design-circular-queue.cpp}

\begin{kaobox}[title=相关题目]
  \begin{itemize}
    \item \href{https://leetcode-cn.com/problems/design-circular-deque/}{设计循环双端队列}
  \end{itemize}
\end{kaobox}

\subsection{\href{https://leetcode.cn/problems/design-front-middle-back-queue/}{设计前中后队列}}

这个题最复杂的在于找到队列的中间位置。我们有三种方式实现该问题:

\begin{itemize}
  \item 使用双向链表,每次要插入中间位置的时候,我们需要遍历链表,找到中间位置。但是这样的复杂度为$O(n)$。
  \item 使用两个双向队列。
  \item 使用一个双向链表,同时使用一个迭代器指向中间位置。这个迭代器伴随着插入和删除操作的进行而移动。
\end{itemize}

我采用了最后一种方法。然而这个方法相当复杂,因为我们需要考虑迭代器的移动。我们通过画图总结出迭代器的移动规律。
首先是当队列为空时,此时无论是插入头部、尾部或者中间,迭代器都应该指向头部。当队列不为空时,我们需要考虑的情况
就相当复杂了。

假设当前队列只有一个元素1,如果此时我们插入尾部元素2。按照定义,我们的迭代器应该继续指向1。如果此时我们再向
尾部插入元素3,此时迭代器应该指向2。因此,我们可以得出一个重要的结论。

\begin{itemize}
  \item 对于插入尾部来说,当元素为奇数个时,不需要移动迭代器。当元素为偶数个时,需要向后移动迭代器。
  \item 对于删除尾部来说,当元素为奇数个时,需要向前移动迭代器。当元素为偶数个时,不需要移动迭代器。
\end{itemize}

\begin{tikzpicture}
  \def\width{1}
  \def\height{1}
  \def\array{{ ,}}

  \draw (0, 0) rectangle ++(\width, \height) node[midway] {1};

  \draw[arrow, color=red] (0.5, 1.5) -- (0.5, 1) node[midway, above left] {middle};
\end{tikzpicture}

\begin{tikzpicture}
  \def\width{1}
  \def\height{1}
  \def\array{{ , , }}

  \draw (0, 0) rectangle ++(\width, \height) node[midway] {1};
  \draw (1, 0) rectangle ++(\width, \height) node[midway] {2};

  \draw[arrow, color=red] (0.5, 1.5) -- (0.5, 1) node[midway, above left] {middle};

\end{tikzpicture}

\begin{tikzpicture}
  \def\width{1}
  \def\height{1}
  \def\array{{ , , }}

  \draw (0, 0) rectangle ++(\width, \height) node[midway] {1};
  \draw (1, 0) rectangle ++(\width, \height) node[midway] {2};
  \draw (2, 0) rectangle ++(\width, \height) node[midway] {3};

  \draw[arrow, color=red] (1.5, 1.5) -- (1.5, 1) node[midway, above left] {middle};

\end{tikzpicture}

当我们插入头部时,情况正好相反。

\begin{itemize}
  \item 对于插入头部来说,当元素为奇数个时,需要向前移动迭代器。当元素为偶数个时,不需要移动迭代器。
  \item 对于删除尾部来说,当元素为奇数个时,不需要移动迭代器。当元素为偶数个时,需要向后移动迭代器。
\end{itemize}

\begin{tikzpicture}
  \def\width{1}
  \def\height{1}
  \def\array{{ ,}}

  \draw (0, 0) rectangle ++(\width, \height) node[midway] {1};

  \draw[arrow, color=red] (0.5, 1.5) -- (0.5, 1) node[midway, above left] {middle};
\end{tikzpicture}

\begin{tikzpicture}
  \def\width{1}
  \def\height{1}
  \def\array{{ , , }}

  \draw (0, 0) rectangle ++(\width, \height) node[midway] {2};
  \draw (1, 0) rectangle ++(\width, \height) node[midway] {1};

  \draw[arrow, color=red] (0.5, 1.5) -- (0.5, 1) node[midway, above left] {middle};

\end{tikzpicture}

\begin{tikzpicture}
  \def\width{1}
  \def\height{1}
  \def\array{{ , , }}

  \draw (0, 0) rectangle ++(\width, \height) node[midway] {3};
  \draw (1, 0) rectangle ++(\width, \height) node[midway] {2};
  \draw (2, 0) rectangle ++(\width, \height) node[midway] {1};

  \draw[arrow, color=red] (1.5, 1.5) -- (1.5, 1) node[midway, above left] {middle};

\end{tikzpicture}

最后,就是插入中间的情况了。我们仍然通过例子来说明,假设目前的队列为$[1,2]$,此时迭代器应该指向1。
如果此时,我们插入中间元素$3$,队列变为$[3,1,2]$,迭代器不应移动。如果此时,我们插入中间元素$4$,队列
变为$[3,4,1,2]$,迭代器应该向前移动。

\begin{itemize}
  \item 对于插入中间来说,当元素为偶数个时,不需要移动迭代器。当元素为奇数个时,需要向前移动迭代器。
  \item 对于删除中间来说,当元素为偶数个时,需要向后移动迭代器。当元素为奇数个时,不需要移动迭代器。
\end{itemize}

于是,我们就可以写出如下的代码。

\lstinputlisting[language=C++]{code/design-front-middle-back-queue.cpp}

\section{复杂数据结构设计}

\section{虚拟场景题目}

\subsection{\href{https://leetcode.cn/problems/design-twitter/}{设计推特}}

首先我们需要考虑用户的数据结构应该怎样设计。每个用户应该有个数据结构用于表征其关注的用户,同时应该有一个
链表用于存放其发送的推文。我们可以使用一个哈希表来存储用户的关注列表,使用链表来存储用户发送的推文。

同时,我们需要一个数据结构来保存推文。我们需要记录推文的id,以及推文的发送时间。当我们需要获取主页时,
我们需要找到用户以及其关注的用户的最近的10条推文。此时,我们就可以把问题转化为合并$k$个有序链表的问题。
我们这次使用优先队列解决这个问题。

\lstinputlisting[language=C++]{code/design-twitter.cpp}

\subsection{\href{https://leetcode.cn/problems/design-browser-history/}{设计浏览器历史记录}}

看似这个题我们需要使用链表操作,实际上我们只需要一个数组加指针移动即可。我们需要一个数组来存储历史记录,
同时我们需要一个指针来记录当前的位置。当我们访问一个新的网页时,我们只需要移动指针,记录当前的值。倒退
网页时,我们只需要移动指针即可。

\lstinputlisting[language=C++]{code/design-browser-history.cpp}

\end{document}
