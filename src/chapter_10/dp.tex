\documentclass[../../main.tex]{subfiles}

\begin{document}

\setchapterpreamble[u]{\margintoc}

\chapter{动态规划}

\section{背包问题}

背包问题需要许多前置知识,本书并不打算详细介绍这些知识,而是直接给出状态转移方程。如果读者对于这些知识
不熟悉,可以参考相关的书籍。

设总共有$n$个物品,其质量为$weight[i], 0 \leq i < n$,价值为$value[i], 0 \leq i < n$。同时
假设背包能够容纳的质量为$w$。我们定义$dp[j]$为背包容量为$j$时的最大价值。

对于0/1背包问题,我们有如下的状态转移方程。

$$
dp[j] = \max(dp[j - weight[i]] + value[i], dp[j])
$$

\begin{lstlisting}[language=C++]
  for (int i = 0; i < n ; i++) {
    for (int j = w; j >= weight[i]; j--) {
      dp[j] = max(dp[j - weight[i]] + value[i], dp[j]);
    }
  }
\end{lstlisting}


对于完全背包问题,我们有如下的状态转移方程。

$$
dp[j] = \max(dp[j - weight[i]] + value[i], dp[j])
$$

\begin{lstlisting}[language=C++]
  for (int i = 0; i < n ; i++) {
    for (int j = weight[i]; j <= w; j++) {
      dp[j] = max(dp[j - weight[i]] + value[i], dp[j]);
    }
  }
\end{lstlisting}

对于完全背包问题,我们需要考虑是先遍历物品还是先遍历背包。如果我们先遍历物品,由于我们人为地确定了
物品的顺序,所以我们得到的永远是组合数。如果我们先遍历背包,那么我们得到的就是排列数。

\subsection{\href{https://leetcode.cn/problems/coin-change/}{零钱兑换}}

由于每种零钱都是无限的,因此这是一个完全背包问题。我们可以很容易地得到状态转移方程。我们设
$dp[j]$为兑换$j$元的最少硬币数。我们可以得到如下的状态转移方程:

$$
dp[j] = \min(dp[j - coins[i]] + 1, dp[j])
$$

\lstinputlisting[language=C++]{code/coin-change.cpp}

\subsection{\href{https://leetcode-cn.com/problems/coin-change-ii/}{零钱兑换II}}
\label{subsec:coin-change-ii}

这个题与上一个题没有任何本质的区别,其区别就在于需要求组合数。我们这样就直接限制物品的遍历顺序即可。
同时我们定义如下的状态。$dp[j]$代表兑换$j$元的组合数。我们可以得到如下的状态转移方程:

$$
dp[j] += dp[j - coins[i]]
$$

\lstinputlisting[language=C++]{code/coin-change-ii.cpp}

\subsection{\href{https://leetcode-cn.com/problems/combination-sum-iv/}{组合总和IV}}

这仍然是一个完全背包问题,但是需要我们求排列数。除此之外与\ref{subsec:coin-change-ii}节中的实现
没有本质区别,由于求排列数我们必须先遍历背包。

\lstinputlisting[language=C++]{code/combination-sum-iv.cpp}

\subsection{\href{https://leetcode.cn/problems/partition-equal-subset-sum/}{分割等和子集}}

这个题看似与0/1背包问题毫无关系,但是我们可以转化为0/1背包问题。我们假设两个集合的和为$sum$,则
如果这两个集合相等。那么每一个集合的和必然都为$sum / 2$。这样我们就可以只对一个集合分析,也就是
对于一个数组,我们可以选择其元素或者不选择其元素。然后判断其容量为$sum / 2$的背包能否存在解。

\lstinputlisting[language=C++]{code/partition-equal-subset-sum.cpp}

\subsection{\href{https://leetcode.cn/problems/target-sum/}{目标和}}

我们实际上可以用DFS解决这个问题,对每一个元素来其有两个选择,一个是加上当前元素,一个是减去当前元素。
然而,我们可以转化为0/1背包问题。我们假设相加的元素为$x$,那么相减的元素为$sum - x$。我们可以得到
$x - (sum - x) = target$。则$x = (target + sum) / 2 $。实际上,我们就转化为了一个0/1背包问题
\sidenote{实际上,当问题仅仅只有两个状态的变化时,我们就可以思考能否应用0/1背包问题}。

同样的,我们就思考从数组中选择元素,能否组成和为$x$的背包。我们定义如下的状态。$dp[j]$代表组成
$j$的组合数。我们可以得到如下的状态转移方程:

$$
dp[j] += dp[j - nums[i]]
$$

\lstinputlisting[language=C++]{code/target-sum.cpp}

\section{打家劫舍系列}

\subsection{\href{https://leetcode-cn.com/problems/house-robber/}{打家劫舍}}

我们可以使用DFS解决这个问题,对于一个房子,我们有两个选择:

\begin{itemize}
  \item 如果我们偷窃了这个房子的前一个房子,我们什么也不能做。
  \item 如果我们没有偷窃这个房子的前一个房子,我们可以偷窃这个房子也可以不偷窃这个房子。
\end{itemize}

根据这样我们就可以写出如下的代码\sidenote{当然,会超时}:

\lstinputlisting[language=C++]{code/house-robber-bfs.cpp}

然而,很明显我们能发现许多重复的计算,我们可以使用记忆化搜索来解决这个问题。然而,我们直接用
状态思考,设$dp[i]$为偷窃前$i$个房子的最大收益。根据上述的选择,我们有两个操作。那么我们
可以得到如下的状态方程:

$$
dp[i] = \max(dp[i - 2] + nums[i], dp[i - 1])
$$

\lstinputlisting[language=C++]{code/house-robber-dp.cpp}

\subsection{\href{https://leetcode.cn/problems/house-robber-ii/}{打家劫舍II}}

本题添加了一个新的条件,也就是房屋成了一个环。我们已经解决了没有环的问题。按照正常的思路,
我们应该需要把这个问题转化为没有环的问题。所以我们可以人为地打破这个环。由于第一个房屋和
最后一个房屋是连接的,所以我们只需要考虑偷窃第一个房屋到第$n - 1$个房屋和第二个房屋到第$n$
个房屋。选择这两种的最大值即可。从而优雅地解决这个问题。

\lstinputlisting[language=C++]{code/house-robber-ii.cpp}

\section{买卖股票问题}

买卖股票问题是状态机DP的典型应用,明确其状态以及状态转移方程是解决这类问题的关键。

\subsection{\href{https://leetcode.cn/problems/best-time-to-buy-and-sell-stock/}{买卖股票的最佳时机}}

我们需要思考状态,在我看来这一步更多的就是去穷举思考。我们只可能位于两个状态即持有股票和未持有股票,则我们很
容易地定义出状态:

\begin{itemize}
  \item $dp[i][0]$表示第$i$天未持有股票的最大收益。
  \item $dp[i][1]$表示第$i$天持有股票的最大收益。
\end{itemize}

由于只能买卖一次,我们能够轻易地得到如下的转移方程:

\begin{align*}
  dp[i][0] &= \max(dp[i - 1][1] + prices[i], dp[i - 1][0]) \\
  dp[i][1] &= \max(dp[i - 1][1], -prices[i])
\end{align*}

同时,我们很容易能够得到初始值:

\begin{align*}
  dp[0][0] &= 0 \\
  dp[0][1] &= -prices[0]
\end{align*}

然后,我们就可以得出如下代码:

\lstinputlisting[language=C++]{code/best-time-to-buy-and-sell-stock.cpp}

\subsection{\href{https://leetcode-cn.com/problems/best-time-to-buy-and-sell-stock-ii/}{买卖股票的最佳时机 II}}

这个题与上一个题没有任何本质的区别,其区别就在于我们可以买卖多次。我们可以很容易地得到如下的转移方程:

\begin{align*}
  dp[i][0] &= \max(dp[i - 1][1] + prices[i], dp[i - 1][0]) \\
  dp[i][1] &= \max(dp[i - 1][1], dp[i - 1][0] - prices[i])
\end{align*}

然后,我们就可以得出如下代码:

\lstinputlisting[language=C++]{code/best-time-to-buy-and-sell-stock-ii.cpp}

\subsection{\href{https://leetcode-cn.com/problems/best-time-to-buy-and-sell-stock-iii/}{买卖股票的最佳时机 III}}

这个题与上一个题没有任何本质的区别,其区别就在于限制了我们买卖的次数。在上两个题中我们维持的状态是持有股票
或者未持有股票。显然,这个题股票的持有状态有5个\sidenote{这是做状态机dp十分重要的思维方式。也就是
穷举出所有的状态,然后去思考状态转移方程}。

\begin{itemize}
  \item 从来没有购买股票
  \item 第一次购买股票
  \item 第一次卖出股票
  \item 第二次购买股票
  \item 第二次卖出股票
\end{itemize}

然后我们定义如下的状态:

\begin{itemize}
  \item $dp[i][0]$表示第$i$天从来没有购买股票的最大收益。
  \item $dp[i][1]$表示第$i$天第一次购买股票的最大收益。
  \item $dp[i][2]$表示第$i$天第一次卖出股票的最大收益。
  \item $dp[i][3]$表示第$i$天第二次购买股票的最大收益。
  \item $dp[i][4]$表示第$i$天第二次卖出股票的最大收益。
\end{itemize}

显然,我们能够得到如下的转移方程\sidenote{实际上这个过程不是显然,考虑状态的转移是要思考的。}:

\begin{align*}
  dp[i][0] &= dp[i - 1][0] \\
  dp[i][1] &= \max(dp[i - 1][1], dp[i - 1][0] - prices[i]) \\
  dp[i][2] &= \max(dp[i - 1][2], dp[i - 1][1] + prices[i], dp[i][0]) \\
  dp[i][3] &= \max(dp[i - 1][3], dp[i - 1][2] - prices[i]) \\
  dp[i][4] &= \max(dp[i - 1][4], dp[i - 1][3] + prices[i], dp[i][2]) \\
\end{align*}

然后,我们就可以非常简单地写出如下的代码了:

\lstinputlisting[language=C++]{code/best-time-to-buy-and-sell-stock-iii.cpp}

\subsection{\href{https://leetcode-cn.com/problems/best-time-to-buy-and-sell-stock-iv/}{买卖股票的最佳时机 IV}}

上一个题仅仅允许两次买卖,这个题允许$k$次买卖。那么我们需要多少个状态呢。显然,我们需要$2k + 1$个状态。
然后我们就可以直接解决这个问题了。我们可以很简单的推导出状态方程。

\begin{align*}
  dp[i][0] &= dp[i - 1][0] \\
  dp[i][j] &= \max(dp[i - 1][j], dp[i - 1][j - 1] - prices[i]), j = 2 * n + 1 \\
  dp[i][j] &= \max(dp[i - 1][j], dp[i - 1][j - 1] + prices[i]), j = 2 * n \\
\end{align*}

然后,我们就可以得出如下代码:

\lstinputlisting[language=C++]{code/best-time-to-buy-and-sell-stock-iv.cpp}

\subsection{\href{https://leetcode-cn.com/problems/best-time-to-buy-and-sell-stock-with-cooldown/}
{最佳买卖股票时机含冷冻期}}

我们可以进行无数次交易了,但是我们需要考虑冷冻期。很显然,我们总共有两大类状态。一类是持有股票或者未持有股票,另
一类是昨日卖出股票或者没有卖出股票。于是我们能够得到以下的状态:

\begin{itemize}
  \item $dp[i][0]$表示第$i$天未持有股票且昨天没卖出股票的最大收益。
  \item $dp[i][1]$表示第$i$天未持有股票且昨天卖出股票的最大收益。
  \item $dp[i][2]$表示第$i$天持有股票且昨天没卖出股票的最大收益。
  \item $dp[i][3]$表示第$i$天持有股票且昨天卖出股票的最大收益(不可能)。
\end{itemize}

然后我们需要思考状态的转换。显然,我们可以得到如下的转移方程\sidenote{
对于$dp[i][0]$来说,其状态必然由第$i-1$天没有持有股票的状态共同决定,
因为昨天不能卖出股票。对于$dp[i][1]$来说,其必须由$dp[i][2]$和今天
购入股票决定。对于$dp[i][2]$来说其可以继续持有,或者在今天由
$dp[i][0]$状态买入股票。
}:

\begin{align*}
  dp[i][0] &= \max(dp[i - 1][0], dp[i - 1][1]) \\
  dp[i][1] &= dp[i - 1][2] + prices[i] \\
  dp[i][2] &= \max(dp[i - 1][2], dp[i - 1][0] - prices[i]) \\
\end{align*}

这样我们就可以得出如下的代码:

\lstinputlisting[language=C++]{code/best-time-to-buy-and-sell-stock-with-cooldown.cpp}

\subsection{\href{https://leetcode.cn/problems/house-robber-iii/}{打家劫舍III}}

你可能会疑惑,为什么这个问题会被归纳为状态机dp问题。实际上,我们能够从状态去思考这个问题,对于一个节点
其只有两个状态:偷窃或者不偷窃。但是我们的状态转移该如何实现了?实际上,我们需要从叶子节点到根节点。
也就是后续遍历。我们可以得到如下的状态转移:

\begin{itemize}
  \item 对于该节点被偷窃的情况,其孩子节点都不能被偷窃。
  \item 对于子节点没有被偷窃的情况,其孩子节点可以被偷窃也可以不被偷窃。
\end{itemize}

根据这两个选择,我们就可以不断进行状态的转移\sidenote{树提供了状态的改变,所以我们不需要手动实现
状态的转换,也就是依赖遍历树本身的操作实现状态转移。}。

\lstinputlisting[language=C++]{code/house-robber-iii.cpp}

\section{经典的一维和二维dp问题}

DP的问题都比较玄学,实际上DP的算法题就是如此的无聊,你难以从一个正常人的思维模式出发去解决这个
问题。所以此部分的题目都是比较经典的题目,也就是说你可能会在面试中遇到这些题目。我并不打算
讲清楚是怎么从常人的思维角度解决问题的\sidenote{因为本质就是猜。}。

\subsection{\href{https://leetcode.cn/problems/longest-increasing-subsequence/}
{最长递增子序列}}

我们仍然去猜,子序列问题是典型的可能会运用到DP的问题。很显然我们直接在$dp[i]$表示以0到$i$的子数组
的最长递增子序列的长度。我们可以得到如下的状态转移方程:

$$
dp[i] = \max (dp[j] + 1, dp[i]), j = 0 \dots i -1, nums[j] < nums[i]
$$

然后我们就能写出如下的代码:

\lstinputlisting[language=C++]{code/longest-increasing-subsequence.cpp}

\subsection{\href{https://leetcode.cn/problems/maximum-subarray/}{最大子序和}}

由于这个题目要求连续的子数组,那么我们必然可以得到一个结论,也就是$dp[i]$必然依赖于$dp[i - 1]$。
我们设$dp[i]$为从0到$i - 1$的子数组的最大子序和。那么我们有两种选择:

\begin{itemize}
  \item 断开当前的子数组,从$i$开始重新计算。
  \item 或者连接当前的子数组。
\end{itemize}

也就是我们需要比较上述两种情况的最大值。我们可以得到如下的状态转移方程:

$$
dp[i] = \max(dp[i - 1] + nums[i], nums[i])
$$

然后我们就能写出如下的代码:

\lstinputlisting[language=C++]{code/maximum-subarray.cpp}

\subsection{\href{https://leetcode.cn/problems/longest-common-subsequence/}{最长公共子序列}}

这个题目是一个典型的二维dp问题。我们设$dp[i][j]$为$A[0 \dots i]$和$B[0 \dots j]$的最长公共子序列。
如果$text1[i] == text2[j]$,那么$dp[i][j] = dp[i - 1][j - 1] + 1$。如果不相等,那么就去寻找
$dp[i - 1][j]$和$dp[i][j - 1]$的最大值\sidenote{最长公共子序列是一个模板题,许多二维dp问题就是
利用该经典问题进行思考的,任何时候都需要有把未知问题转化为已知问题的思维模式}。

\lstinputlisting[language=C++]{code/longest-common-subsequence.cpp}

\begin{kaobox}[title=相似题目]
  解决了这个问题,我们就能够直接解决\href{https://leetcode.cn/problems/uncrossed-lines/}{不相交的线}。
  因为本质不相交的线就是最长公共子序列。
\end{kaobox}

\subsection{\href{https://leetcode.cn/problems/maximum-length-of-repeated-subarray/}{最长重复子数组}}

这个题目与上一个题目没有任何本质的区别,其区别就在于序列必须连续,那么$dp[i][j]$的状态必然
只能由$dp[i - 1][j - 1]$确定。

\lstinputlisting[language=C++]{code/maximum-length-of-repeated-subarray.cpp}

\section{最长公共子序列的应用}

这一章涉及到了最长公共子序列的应用,其代码都是类似的,主要着重于递归公式的推导。

\subsection{\href{https://leetcode.cn/problems/distinct-subsequences/}{不同的子序列}}

对于这个问题,我们直接套用公共子序列的模板。如果当前的字符相等即$s[i] = t[j]$,那么我们可以选择删除
当前的字符,即$dp[i][j - 1]$。也可以选择不删除$dp[i - 1][j -1]$。如果不相等,我们只能选择删除
当前的字符,即$dp[i - 1][j]$。我们可以得到如下的状态转移方程:

$$
dp[i][j] = \begin{cases}
  dp[i - 1][j - 1] + dp[i][j - 1], & s[i] = t[j] \\
  dp[i - 1][j], & s[i] \neq t[j]
\end{cases}
$$

\subsection{\href{https://leetcode.cn/problems/delete-operation-for-two-strings/}{两个字符串的删除操作}}

既然要求删除的次数最少,本质实际上也是在求最长公共子序列。然而,我们不这样做。我们仍然思考。如果当前的字符
相等即$s[i] = t[j]$,那么我们肯定不删,即$dp[i - 1][j - 1]$。如果不相等,我们只能选择删除两个
字符$dp[i - 1][j - 1] + 2$,删除$s[i]$,即$dp[i - 1][j] + 1$或者删除$t[j]$即$dp[i][j - 1] + 1$。

然后,我们可以得到如下的递推公式\sidenote{实际上$dp[i - 1][j - 1] + 2$可以省略掉。}:

$$
dp[i][j] = \begin{cases}
  dp[i - 1][j - 1] , & s[i] = t[j] \\
  \min(dp[i - 1][j] + 1, dp[i][j - 1] + 1), & s[i] \neq t[j]
\end{cases}
$$

\subsection{\href{https://leetcode.cn/problems/edit-distance/}{编辑距离}}

如果当前的字符相等即$s[i] = t[j]$,那么我们肯定啥也不操作,即$dp[i - 1][j -1]$。如果不相等,那么我们
有三个选择(删除、添加和替换)。这个最重要的是必须意识到添加和删除没有任何本质的区别。我们删除了一个元素,
本质就是给另一个字符串添加了一个元素。

\section{一维dp问题}

\subsection{\href{https://leetcode.cn/problems/decode-ways/}{解码方法}}

解决dp问题的思路永远是猜。我们设$dp[i]$为长度为$i$的子序列的解码个数,显然其由两个子问题构成,也就是
$dp[i - 2]$和$dp[i - 1]$。如果$s_{i -1}s_{i}$能够解码,则$dp[i] += dp[i - 2]$,如果
$s_{i}$能够解码,则有$dp[i] += dp[i - 1]$。我们可以得到如下的递推公式:

$$
dp[i] += \begin{cases}
  dp[i - 2], & s_{i - 1}s_{i} \in [10, 26] \\
  dp[i - 1], & s_{i} \in [1, 9]
\end{cases}
$$

然后,我们就可以得到如下的代码:

\lstinputlisting[language=C++]{code/decode-ways.cpp}

\subsection{\href{https://leetcode.cn/problems/count-ways-to-build-good-strings/}
{统计构造好字符串的方案数}}

仍然是猜,猜$dp[i]$为长度为$i$的字符串的构建方式有多少种,我们能够得到如下的转移方程

\begin{align*}
  dp[i] &= (dp[i] + dp[i - one]) \% M \\
  dp[i] &= (dp[i] + dp[i - zero]) \% M
\end{align*}

然后,我们就能得出如下的代码:

\lstinputlisting[language=C++]{code/count-ways-to-build-good-strings.cpp}

\section{复杂的状态机dp}

\subsection{\href{https://leetcode.cn/problems/student-attendance-record-ii/}{学生出勤记录 II}}

这个题相当复杂,题目给了我们两个条件,一个是$A$必须严格少于两天,另一个是不存在连续的3个$L$。那么我们就可以
穷举出所有的状态了:

\begin{itemize}
  \item $dp[i][0]$: 有一个$A$和两个末尾连续的$L$。
  \item $dp[i][1]$: 有一个$A$和一个位于末尾的$L$。
  \item $dp[i][2]$: 没有$A$和两个末尾连续的$L$。
  \item $dp[i][3]$: 没有$A$和一个位于末尾的$L$。
  \item $dp[i][4]$: 有一个$A$及末尾不是$L$。
  \item $dp[i][5]$: 没有$A$及末尾不是$L$。
\end{itemize}

我们从$i - 1$的状态考虑问题:

\begin{itemize}
  \item $dp[i-1][0]$只能添加$P$成为$dp[i][4]$。
  \item $dp[i-1][1]$能添加$L$成为$dp[i][0]$,或者添加$P$成为$dp[i][4]$。
  \item $dp[i-1][2]$能添加$A$成为$dp[i][4]$, 或者添加$P$成为$dp[i][5]$。
  \item $dp[i-1][3]$能添加$L$成为$dp[i][2]$,添加$A$成为$dp[i][4]$,
  或者添加$P$成为$dp[i][5]$。
  \item $dp[i-1][4]$能添加$L$成为$dp[i][1]$,或者添加$P$成为$dp[i][4]$。
  \item $dp[i-1][5]$能添加$A$成为$dp[i][4]$,添加$L$成为$dp[i][3]$,
  或者添加$P$成为$dp[i][5]$。
\end{itemize}

这样我们就能解决这个问题了。

\lstinputlisting[language=C++]{code/student-attendance-record-ii.cpp}

\end{document}
