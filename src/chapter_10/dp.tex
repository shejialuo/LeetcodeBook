\documentclass[../../main.tex]{subfiles}

\begin{document}

\setchapterpreamble[u]{\margintoc}

\chapter{动态规划}

\section{买卖股票问题}

买卖股票问题是状态机DP的典型应用,明确其状态以及状态转移方程是解决这类问题的关键。

\subsection{\href{https://leetcode.cn/problems/best-time-to-buy-and-sell-stock/}{买卖股票的最佳时机}}

我们需要思考状态,在我看来这一步更多的就是去穷举思考。我们只可能位于两个状态即持有股票和未持有股票,则我们很
容易地定义出状态:

\begin{itemize}
  \item $dp[i][0]$表示第$i$天未持有股票的最大收益。
  \item $dp[i][1]$表示第$i$天持有股票的最大收益。
\end{itemize}

由于只能买卖一次,我们能够轻易地得到如下的转移方程:

\begin{align*}
  dp[i][0] &= \max(dp[i - 1][1] + prices[i], dp[i - 1][0]) \\
  dp[i][1] &= \max(dp[i - 1][1], -prices[i])
\end{align*}

同时,我们很容易能够得到初始值:

\begin{align*}
  dp[0][0] &= 0 \\
  dp[0][1] &= -prices[0]
\end{align*}

然后,我们就可以得出如下代码:

\lstinputlisting[language=C++]{code/best-time-to-buy-and-sell-stock.cpp}

\subsection{\href{https://leetcode-cn.com/problems/best-time-to-buy-and-sell-stock-ii/}{买卖股票的最佳时机 II}}

这个题与上一个题没有任何本质的区别,其区别就在于我们可以买卖多次。我们可以很容易地得到如下的转移方程:

\begin{align*}
  dp[i][0] &= \max(dp[i - 1][1] + prices[i], dp[i - 1][0]) \\
  dp[i][1] &= \max(dp[i - 1][1], dp[i - 1][0] - prices[i])
\end{align*}

然后,我们就可以得出如下代码:

\lstinputlisting[language=C++]{code/best-time-to-buy-and-sell-stock-ii.cpp}

\subsection{\href{https://leetcode-cn.com/problems/best-time-to-buy-and-sell-stock-iii/}{买卖股票的最佳时机 III}}

这个题与上一个题没有任何本质的区别,其区别就在于限制了我们买卖的次数。在上两个题中我们维持的状态是持有股票
或者未持有股票。显然,这个题股票的持有状态有5个\sidenote{这是做状态机dp十分重要的思维方式。也就是
穷举出所有的状态,然后去思考状态转移方程}。

\begin{itemize}
  \item 从来没有购买股票
  \item 第一次购买股票
  \item 第一次卖出股票
  \item 第二次购买股票
  \item 第二次卖出股票
\end{itemize}

然后我们定义如下的状态:

\begin{itemize}
  \item $dp[i][0]$表示第$i$天从来没有购买股票的最大收益。
  \item $dp[i][1]$表示第$i$天第一次购买股票的最大收益。
  \item $dp[i][2]$表示第$i$天第一次卖出股票的最大收益。
  \item $dp[i][3]$表示第$i$天第二次购买股票的最大收益。
  \item $dp[i][4]$表示第$i$天第二次卖出股票的最大收益。
\end{itemize}

显然,我们能够得到如下的转移方程\sidenote{实际上这个过程不是显然,考虑状态的转移是要思考的。}:

\begin{align*}
  dp[i][0] &= dp[i - 1][0] \\
  dp[i][1] &= \max(dp[i - 1][1], dp[i - 1][0] - prices[i]) \\
  dp[i][2] &= \max(dp[i - 1][2], dp[i - 1][1] + prices[i], dp[i][0]) \\
  dp[i][3] &= \max(dp[i - 1][3], dp[i - 1][2] - prices[i]) \\
  dp[i][4] &= \max(dp[i - 1][4], dp[i - 1][3] + prices[i], dp[i][2]) \\
\end{align*}

然后,我们就可以非常简单地写出如下的代码了:

\lstinputlisting[language=C++]{code/best-time-to-buy-and-sell-stock-iii.cpp}

\subsection{\href{https://leetcode-cn.com/problems/best-time-to-buy-and-sell-stock-iv/}{买卖股票的最佳时机 IV}}

上一个题仅仅允许两次买卖,这个题允许$k$次买卖。那么我们需要多少个状态呢。显然,我们需要$2k + 1$个状态。
然后我们就可以直接解决这个问题了。我们可以很简单的推导出状态方程。

\begin{align*}
  dp[i][0] &= dp[i - 1][0] \\
  dp[i][j] &= \max(dp[i - 1][j], dp[i - 1][j - 1] - prices[i]), j = 2 * n + 1 \\
  dp[i][j] &= \max(dp[i - 1][j], dp[i - 1][j - 1] + prices[i]), j = 2 * n \\
\end{align*}

然后,我们就可以得出如下代码:

\lstinputlisting[language=C++]{code/best-time-to-buy-and-sell-stock-iv.cpp}

\subsection{\href{https://leetcode-cn.com/problems/best-time-to-buy-and-sell-stock-with-cooldown/}
{最佳买卖股票时机含冷冻期}}

我们可以进行无数次交易了,但是我们需要考虑冷冻期。很显然,我们总共有两大类状态。一类是持有股票或者未持有股票,另
一类是昨日卖出股票或者没有卖出股票。于是我们能够得到以下的状态:

\begin{itemize}
  \item $dp[i][0]$表示第$i$天未持有股票且昨天没卖出股票的最大收益。
  \item $dp[i][1]$表示第$i$天未持有股票且昨天卖出股票的最大收益。
  \item $dp[i][2]$表示第$i$天持有股票且昨天没卖出股票的最大收益。
  \item $dp[i][3]$表示第$i$天持有股票且昨天卖出股票的最大收益(不可能)。
\end{itemize}

然后我们需要思考状态的转换。显然,我们可以得到如下的转移方程\sidenote{
对于$dp[i][0]$来说,其状态必然由第$i-1$天没有持有股票的状态共同决定,
因为昨天不能卖出股票。对于$dp[i][1]$来说,其必须由$dp[i][2]$和今天
购入股票决定。对于$dp[i][2]$来说其可以继续持有,或者在今天由
$dp[i][0]$状态买入股票。
}:

\begin{align*}
  dp[i][0] &= \max(dp[i - 1][0], dp[i - 1][1]) \\
  dp[i][1] &= dp[i - 1][2] + prices[i] \\
  dp[i][2] &= \max(dp[i - 1][2], dp[i - 1][0] - prices[i]) \\
\end{align*}

这样我们就可以得出如下的代码:

\lstinputlisting[language=C++]{code/best-time-to-buy-and-sell-stock-with-cooldown.cpp}

\section{复杂的状态机dp}

\subsection{\href{https://leetcode.cn/problems/student-attendance-record-ii/}{学生出勤记录 II}}

这个题相当复杂,题目给了我们两个条件,一个是$A$必须严格少于两天,另一个是不存在连续的3个$L$。那么我们就可以
穷举出所有的状态了:

\begin{itemize}
  \item $dp[i][0]$: 有一个$A$和两个末尾连续的$L$。
  \item $dp[i][1]$: 有一个$A$和一个位于末尾的$L$。
  \item $dp[i][2]$: 没有$A$和两个末尾连续的$L$。
  \item $dp[i][3]$: 没有$A$和一个位于末尾的$L$。
  \item $dp[i][4]$: 有一个$A$及末尾不是$L$。
  \item $dp[i][5]$: 没有$A$及末尾不是$L$。
\end{itemize}

我们从$i - 1$的状态考虑问题:

\begin{itemize}
  \item $dp[i-1][0]$只能添加$P$成为$dp[i][4]$。
  \item $dp[i-1][1]$能添加$L$成为$dp[i][0]$,或者添加$P$成为$dp[i][4]$。
  \item $dp[i-1][2]$能添加$A$成为$dp[i][4]$, 或者添加$P$成为$dp[i][5]$。
  \item $dp[i-1][3]$能添加$L$成为$dp[i][2]$,添加$A$成为$dp[i][4]$,
  或者添加$P$成为$dp[i][5]$。
  \item $dp[i-1][4]$能添加$L$成为$dp[i][1]$,或者添加$P$成为$dp[i][4]$。
  \item $dp[i-1][5]$能添加$A$成为$dp[i][4]$,添加$L$成为$dp[i][3]$,
  或者添加$P$成为$dp[i][5]$。
\end{itemize}

这样我们就能解决这个问题了。

\lstinputlisting[language=C++]{code/student-attendance-record-ii.cpp}

\end{document}
