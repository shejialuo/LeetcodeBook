\documentclass[../../main.tex]{subfiles}

\begin{document}

\setchapterpreamble[u]{\margintoc}

\chapter{二分查找}

二分查找是一个相当重要的算法,其实现并不算复杂,然而当二分查找运用于解决实际问题中时,其实现的细节却往往不那么容易。
本章将介绍二分查找的基本思想,以及如何将二分查找应用于解决实际问题中。

\section{基础}

\subsection{\href{https://leetcode.cn/problems/binary-search/}{二分查找}}

只需要把区别维持好就行了,我维持的区间是左闭右开。

\lstinputlisting[language=C++]{code/binary-search.cpp}

\subsection{\href{https://leetcode.cn/problems/find-first-and-last-position-of-element-in-sorted-array/}
{在排序数组中查找元素的第一个和最后一个位置}}

本题考察的仍然是二分查找的基础操作,其涉及到\texttt{lower\_bound}和\texttt{upper\_bound}两个操作。
首先我们考虑如何查找到第一个位置,当我们访问到目标值时,我们应该继续向左移动继续搜索。对于最后一个位置,当
我们访问到目标值时,我们应该继续向右移动继续搜索。

\lstinputlisting[language=C++]{code/find-first-and-last-position-of-element-in-sorted-array.cpp}

\section{应用}

二分法的应用场景很多,实际上更多的是二分是一种思想。利用二分解决问题的关键问题在于必须明确二分法是正确的。

\subsection{\href{https://leetcode.cn/problems/search-in-rotated-sorted-array/}{搜索旋转排序数组}}

本题是二分查找的一个变形。首先我们设这个有序数组为$a_{1}a_{2}a_{j}\dots a_{n}$,我们进行
旋转,此时数组成为$a_{j + 1} \dots a_{n} a_{1}a_{2} \dots a_{j}$。很显然数组被分为了
两个有序数组。显然,我们需要区分目标值在哪个数组中,然后再进行二分查找。

我们通过举例子来说明这个问题。假设数组为$2,3,4,5,6,7,0,1$,我们的目标值为$1$\sidenote{
显然,举了这个例子我们就能反推$7,8,1,2,3,4,5,6$该如何寻找目标值为$8$的情况。
}。

\begin{itemize}
  \item $start = 0, end = 8, mid = 4$。我们找到了当前的中间值为6。其大于目标值,按照正常
  的二分搜索,我们似乎需要移动$end$指针。但是由于目标值小于2且当前的中间值大于2,证明我们位于
  第一个数组中,我们需要移动指针到第二个数组。因此我们反而应该移动$start$指针。
  \item $start = 5, end = 8, mid = 6$。我们找到了当前的中间值为0。其小于目标值。我们此时
  可以正常的按照二分搜索移动$start$指针。
  \item $start = 7, end = 8, mid = 7$。我们找到了1。
\end{itemize}

于是我们可以得出如下的总结:

\begin{itemize}
  \item 当中间值大于目标值时,有两种情况。目标值可能位于第一个数组中,也可能位于第二个数组中。
  如果说中间值位于第一个数组中(即$nums[mid] >= nums[0]$)且目标值位于第一个数组中(
  $target >= nums[0]$),那么我们可以遵循正常的二分搜索。如果目标值位于第二个数组中,那么我们
  需要移动$start$指针。
  \item 当中间值小于目标值时,有两种情况。目标值可能位于第一个数组中,也可能位于第二个数组中。
  如果说中间值位于第二个数组中(即$nums[mid] < nums[0]$)且目标值位于第二个数组中(
  $target < nums[0]$),那么我们可以遵循正常的二分搜索。如果目标值位于第一个数组中,那么我们
  需要移动$end$指针。
\end{itemize}

然后,我们可以得出如下的代码\sidenote{实际上,二分查找的应用是极其困难的。}。

\lstinputlisting[language=C++]{code/search-in-rotated-sorted-array.cpp}

\subsection{\href{https://leetcode.cn/problems/first-bad-version/}{第一个错误的版本}}

这个题是很明显的二分应用,由于题目中准确的表示出了只要当前的版本是错误的,那么后面的版本是一定
是错误的。我们已经在排序数组中查找元素的第一个和最后一个位置问题中阐述了这个思想。

\lstinputlisting[language=C++]{code/first-bad-version.cpp}

\subsection{\href{https://leetcode.cn/problems/search-a-2d-matrix/}{搜索二维矩阵}}

我们可以明显发现矩阵有两个有序状态

\begin{itemize}
  \item 每一行都是有序的。
  \item 第$n$行的每一个元素都小于$n + 1$行。
\end{itemize}

根据这个状态,当我们需要搜索一个数据的时候,我们可以先使用二分搜索找到其所在的行,然后对其
所在的行使用二分搜索查找元素。

\lstinputlisting[language=C++]{code/search-a-2d-matrix.cpp}

\end{document}
