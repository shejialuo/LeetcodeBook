\documentclass[../../main.tex]{subfiles}

\begin{document}

\chapter{双指针}

双指针是在算法中考察的很多的一类问题,其依赖于两个指针互相移动从而解决问题。
指针的移动方向可以相同,也可以相反,也可以是一前一后,这些都是根据具体问题而定的。

在我看来,这上面的话都是废话。并没有告诉你应该如何去实际的解决一个问题。这个章节主要
就是根据实际的问题来思考该如何使用双指针。

\section{N数和}

\subsection{\href{https://leetcode-cn.com/problems/two-sum/}{两数之和}}

几乎所有解决这个问题的方法都是使用哈希表,但是这里我们使用双指针来解决这个问题。
当我们解决了这个问题,我们才能继续地思考三数之和,四数之和等等问题。

这个题的暴力解法是显而易见的。如果要使用双指针解决这个问题,我们的做法就是排序,
然后使用两个指针分别指向数组的头和尾,然后根据两个指针指向的元素的和与目标值的
大小关系来移动指针。

在这个过程中,我们可以确定一个循环不变量:\textbf{在数组中,左侧指针之前的所有元素都小于等于
右侧指针之后的所有元素}。对指针\verb|start|和\verb|end|,我们可以得到其值

$$
sum = nums[start] + nums[end]
$$

如果$sum$小于目标值,那么我们就需要增大$sum$,所以我们需要增大\verb|start|,反之亦然。
我们可以简单地写出如下的代码:

\lstinputlisting[language=C++]{code/two_sum.cpp}

\end{document}
