\documentclass[../../main.tex]{subfiles}

\begin{document}

\setchapterpreamble[u]{\margintoc}

\chapter{链表}

链表是最为基本的数据结构,其题目求解过程无非在于画图寻找规律,确定好循环不变量,
进而求解问题。

\section{双指针解决链表问题}

双指针能解决许多链表中的经典问题,比如链表的中点,链表的倒数第k个节点等等。其
核心的思路在于,链表无法像数组一样通过下标访问。故通过确定距离或者使用快慢指针
来解决问题。

\subsection{\href{https://leetcode.cn/problems/middle-of-the-linked-list/}{链表的中间结点}}

最简单的方法,莫过于遍历链表,得到链表的长度,然后再遍历一次,找到中间节点。然而
我们可以使用两个指针,一个为\verb|slow|,一个为\verb|fast|,\verb|slow|每次
走一步,\verb|fast|每次走两步,当\verb|fast|走到链表尾部时,\verb|slow|刚好
位于链表的中间位置。

当链表的长度是奇数时,其过程如下图所示。

\begin{itemize}
  \item
    \begin{tikzpicture}[node distance=2cm]
      \node (node1) [node] {Node 1};
      \node (node2) [node, right of=node1] {Node 2};
      \node (node3) [node, right of=node2] {Node 3};


      \coordinate (slow) at (0,1);
      \coordinate (fast) at (0.2,1);

      \draw [arrow] (slow) -- (node1.north) node[midway,above left] {slow};
      \draw [arrow] (fast) -- (0.2,0.24) node[midway,above right] {fast};

      \draw [arrow] (node1) -- (node2);
      \draw [arrow] (node2) -- (node3);

    \end{tikzpicture}
  \item
    \begin{tikzpicture}[node distance=2cm]
      \node (node1) [node] {Node 1};
      \node (node2) [node, right of=node1] {Node 2};
      \node (node3) [node, right of=node2] {Node 3};

      \coordinate (slow) at (2,1);
      \coordinate (fast) at (4,1);

      \draw [arrow] (slow) -- (node2.north) node[midway,above left] {slow};
      \draw [arrow] (fast) -- (node3.north) node[midway,above right] {fast};

      \draw [arrow] (node1) -- (node2);
      \draw [arrow] (node2) -- (node3);

    \end{tikzpicture}
  \item
    \begin{tikzpicture}[node distance=2cm]
      \node (node1) [node] {Node 1};
      \node (node2) [node, right of=node1] {Node 2};
      \node (node3) [node, right of=node2] {Node 3};

      \coordinate (slow) at (2,1);
      \coordinate (fast) at (5,1);

      \draw [arrow] (slow) -- (node2.north) node[midway,above left] {slow};
      \draw [arrow] (fast) -- (5, 0) node[midway,above right] {fast};

      \draw [arrow] (node1) -- (node2);
      \draw [arrow] (node2) -- (node3);

    \end{tikzpicture}
\end{itemize}

当链表的长度是偶数时,其过程如下图所示。

\begin{itemize}
  \item
    \begin{tikzpicture}[node distance=2cm]
      \node (node1) [node] {Node 1};
      \node (node2) [node, right of=node1] {Node 2};
      \node (node3) [node, right of=node2] {Node 3};
      \node (node4) [node, right of=node3] {Node 4};

      \coordinate (slow) at (0,1);
      \coordinate (fast) at (0.2,1);

      \draw [arrow] (slow) -- (node1.north) node[midway,above left] {slow};
      \draw [arrow] (fast) -- (0.2,0.24) node[midway,above right] {fast};

      \draw [arrow] (node1) -- (node2);
      \draw [arrow] (node2) -- (node3);
      \draw [arrow] (node3) -- (node4);
    \end{tikzpicture}
  \item
    \begin{tikzpicture}[node distance=2cm]
      \node (node1) [node] {Node 1};
      \node (node2) [node, right of=node1] {Node 2};
      \node (node3) [node, right of=node2] {Node 3};
      \node (node4) [node, right of=node3] {Node 4};

      \coordinate (slow) at (2,1);
      \coordinate (fast) at (4,1);

      \draw [arrow] (slow) -- (node2.north) node[midway,above left] {slow};
      \draw [arrow] (fast) -- (node3.north) node[midway,above right] {fast};

      \draw [arrow] (node1) -- (node2);
      \draw [arrow] (node2) -- (node3);
      \draw [arrow] (node3) -- (node4);

    \end{tikzpicture}
  \item
    \begin{tikzpicture}[node distance=2cm]
      \node (node1) [node] {Node 1};
      \node (node2) [node, right of=node1] {Node 2};
      \node (node3) [node, right of=node2] {Node 3};
      \node (node4) [node, right of=node3] {Node 4};

      \coordinate (slow) at (4,1);
      \coordinate (fast) at (7,1);

      \draw [arrow] (slow) -- (node3.north) node[midway,above left] {slow};
      \draw [arrow] (fast) -- (7, 0) node[midway,above right] {fast};

      \draw [arrow] (node1) -- (node2);
      \draw [arrow] (node2) -- (node3);
      \draw [arrow] (node3) -- (node4);

    \end{tikzpicture}
\end{itemize}

通过上述的描述,我们已经可以解决这个问题,给出如下的代码。\sidenote{实际上,对于这个问题,我们可以给出严格
的数学证明。无论链表的长度是奇数还是偶数,$slow$指针最多移动$\rfloor n / 2 \rfloor$次。
当$n$为偶数时,$fast$指针需要移动$n - 1$个元素,$n - 1$必为奇数,故其必然会在最后一次循环
移动到链表的倒数第二个元素上。此时其移动了$n / 2 - 1$次,然后进行最后一次循环,$slow$也会移动,故
总共移动$ (n - 1) / 2$次。当$n$为奇数时,$fast$指针需要移动$n - 1$个元素,$n - 1$必然为偶数,故其必然会在最后一次循环
移动到链表的最后一个元素上。然后进行最后一次循环,$slow$不会移动,故总共移动$(n - 1) / 2$次。}

\lstinputlisting[language=C++]{code/middle-of-the-linked-list.cpp}

\subsection{\href{https://leetcode.cn/problems/remove-nth-node-from-end-of-list/}{删除链表的倒数第N个节点}}

首先,要删除第$N$个节点,我们需要找到第$N - 1$个节点,然后将其指向第$N + 1$个节点。如果我们要完成
一次遍历就要解决这个问题,那么在完成一次遍历后的终止条件是什么?也就是遇到尾节点(当然也可以是遇到
空指针)。我们反推这个过程,假设我有一个指针已经指向了尾节点,我希望同时有个指针指向第$N - 1$个节点。
那么这两个指针的距离必然为$N$。

于是,我们假设这两个指针同时倒退,直到一个指针遇到头节点。那么把这个过程反过来,就是我们的解决方案。
\sidenote{虽然这是一个简单的问题,但是在我看来这是一个思维方式,因为程序本质也是状态,当我们确定
结束时的状态后,可以利用该信息去思考如何构造程序。}

\begin{kaobox}[title=边界条件]
  然而,我们必须思考一个边界条件,如果我们要删除头节点,我们该怎么办?解决这个问题最好的方法并不是
  单独处理头节点,而是在头节点前面加一个虚拟节点,这样我们就可以统一处理了。
\end{kaobox}

\lstinputlisting[language=C++]{code/remove-nth-node-from-end-of-list.cpp}

\subsection{环形链表问题}

判断一个链表是否有环,涉及到了两个问题,\href{https://leetcode-cn.com/problems/linked-list-cycle/}
{环形链表}和\href{https://leetcode-cn.com/problems/linked-list-cycle-ii/}{环形链表II}。这两个
问题的解决思路是一样的,所以此处进行讲解。

对于判断链表是否有环,我们可以采用快慢指针的方法。我们让两个指针同时从头节点出发,快指针每次移动两步,慢指针
每次移动一步。如果链表有环,那么快指针必然会追上慢指针。如果链表没有环,那么快指针会先到达尾节点。\sidenote{
这就和在操场里跑步一个道理}

根据上述的思路,我们就可以得出如下的代码:

\lstinputlisting[language=C++]{code/linked-list-cycle.cpp}

然而,我们应该如何找到环的入口呢?作为一个从来没有刷过这个问题的人,很难通过数学证明的方式得出当快慢指针相遇时,
将慢指针移动到头节点,然后快慢指针同时移动,当两个指针相遇时,就是环的入口。\sidenote{
  要证明这个结论,还是相当简单的,假设头节点到环的入口的距离为$x$,环的入口到相遇点的距离为$y$,环的大小为$l$。
  假设慢指针移动了$s$圈,快指针移动了$f$圈。那么我们可以得到如下的等式:
  \begin{align*}
  x + fl + y &= 2 * (x + sl + y) \\
  x + y &= (f - 2s)l \\
  y &= kl - x
  \end{align*}
  然后我们将慢指针移动到头节点,快慢指针同时移动,当两个指针相遇时,快指针正好移动了$k$圈,也就是环的入口。
}

如果是按照正常的思维该怎么去做这个问题呢?我们可以用一个哈希表来存储已经访问过的节点,当我们访问到一个节点时,
我们就去哈希表中查找是否存在该节点,如果存在,那么就是环的入口,如果不存在,那么就将该节点加入到哈希表中。
\sidenote{这不是最高效的做法,但这是最符合思维的做法。}

\lstinputlisting[language=C++]{code/linked-list-cycle-ii.cpp}

\section{链表基础操作}

链表的基础操作有许多,例如反转链表,合并有序链表等。通过这些基础的操作,我们可以去解决更加复杂的问题。

\subsection{\href{https://leetcode.cn/problems/reverse-linked-list/}{反转链表}}

如何去反转一个链表呢?最粗暴的方式就是遍历链表然后使用一个数组记录链表的值,最后使用头插法构建一个
新的链表即可。

然而,我们可以更加聪明一点,链表的问题解决方式用于是画图,并确定循环不变量。\sidenote{可能许多题目的解法
都不会告诉你去确定循环不变量,然而确定循环不变量是一个很重要的步骤,是保证我们写代码时不出错的一个很关键
要素。}

\begin{itemize}
  \item
    \begin{tikzpicture}[node distance=2cm]
      \node (node1) [node] {Node 1};
      \node (node2) [node, right of=node1] {Node 2};
      \node (node3) [node, right of=node2] {Node 3};

      \coordinate (prev) at (0,1);
      \coordinate (cur) at (2,1);

      \draw [arrow] (prev) -- (node1.north) node[midway,above left] {prev};
      \draw [arrow] (cur) -- (node2.north) node[midway,above right] {cur};

      \draw [arrow] (node1) -- (node2);
      \draw [arrow] (node2) -- (node3);

    \end{tikzpicture}
  \item
  \begin{tikzpicture}[node distance=2cm]
    \node (node1) [node] {Node 1};
    \node (node2) [node, right of=node1] {Node 2};
    \node (node3) [node, right of=node2] {Node 3};

    \coordinate (prev) at (2,1);
    \coordinate (cur) at (4,1);

    \draw [arrow] (prev) -- (node2.north) node[midway,above left] {prev};
    \draw [arrow] (cur) -- (node3.north) node[midway,above right] {cur};

    \draw [arrow] (node2) -- (node1);
    \draw [arrow] (node2) -- (node3);

  \end{tikzpicture}
\end{itemize}

通过上述分析可知,我们可以确定循环不变量为在遍历过程中,已反转部分的链表一定是\verb|prev|指针所指
向的链表,而当前节点一定是\verb|cur|指针所指向的节点:初始时,已反转部分为$nullptr$,
那么$prev = nullptr$。\sidenote{我们其实通过循环不变量确定了一个十分关键的要素,也就是$prev$的
初始值。}。故我们可以给出如下的代码:

\lstinputlisting[language=C++]{code/reverse-linked-list.cpp}

\end{document}
