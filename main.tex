\documentclass{kaobook}

\usepackage{zhnumber}
\usepackage{xeCJK}
\usepackage{subfiles}
\usepackage{listings}

\usepackage[framed=true]{kaotheorems}

\makenomenclature

\lstset{
  language=C++,
  basicstyle=\ttfamily\small,
  keywordstyle=\color{blue},
  commentstyle=\color{gray},
  stringstyle=\color{purple},
  showstringspaces=false,
  numbers=left,
  numberstyle=\tiny\color{gray},
  tabsize=2,
  breaklines=true,
  breakatwhitespace=true
}

% Use your own fonts
\setCJKmainfont{Noto Sans CJK SC}
\setCJKmonofont{Noto Sans Mono CJK SC}
\setCJKsansfont{Noto Sans CJK SC}

\begin{document}

\renewcommand\contentsname{目录}
\renewcommand\listfigurename{插图}
\renewcommand\listtablename{表格}
\renewcommand\indexname{索引}
\renewcommand\bibname{参考文献}
\renewcommand\figurename{图}
\renewcommand\tablename{表}

\titlehead{献给我自己刷题的时光}
\title{Leetcode Book:个人的刷题笔记}
\author[佘嘉洛]{佘嘉洛\thanks{一个一直寻找着救赎之路的普通人}}
\date{\today}

\frontmatter

\makeatletter
\uppertitleback{\@titlehead} % Header

\lowertitleback{
	\textbf{开源免费}\\
  本书完全开源和免费,只要你遵守下方的协议,你可以自由地使用本书的内容。

  如果你发现任何的问题,请向仓库提交issue或者pull request。
  \\\url{https://github.com/shejialuo/LeetcodeBook}

  \medskip

  \textbf{协议}\\
  \href{http://creativecommons.org/licenses/by-nc-sa/4.0/}{Creative Commons Attribution-NonCommercial-ShareAlike 4.0
  International License.}

  \medskip

  \textbf{鸣谢} \\
  十分感谢\LaTeX 模板作者\href{fmarotta}{https://github.com/fmarotta/}

  其开源代码位于
  \\\url{https://github.com/fmarotta/kaobook}
  \medskip

}
\makeatother

%----------------------------------------------------------------------------------------
%	DEDICATION
%----------------------------------------------------------------------------------------

\dedication{
  献给我的母亲和父亲,感谢他们给予我的一切。
}

\maketitle

\subfile{src/preface/preface.tex}
\index{perface}

\begingroup

\setlength{\textheight}{230\hscale}

\etocstandarddisplaystyle
\etocstandardlines
\tableofcontents

\endgroup

\mainmatter

\setchapterstyle{kao}

\subfile{src/chapter_1/two_pointer.tex}

\end{document}